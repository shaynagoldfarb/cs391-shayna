\documentclass[11pt]{article}
\usepackage{latexsym}
\oddsidemargin=-0.5cm
\evensidemargin=-0.5cm
\textwidth=17.60cm
\textheight=23cm
\topmargin=-2cm
\headsep=1cm

\title{Modern Compiler Design and Implementation\break CAS CS 391X1 Summer 2025}
\author{(Syllabus)}
\date{}

\begin{document}
\maketitle
\thispagestyle{empty}

\begin{itemize}
\item {\bf Semester} Summer 2025
\item {\bf Teaching Staff}
\begin{itemize}
\item
Instructor -- Hongwei Xi
\begin{itemize}
\item
Office Hours: TBA
\item
Location: CDS 727, e-mail: \texttt{hwxi@cs.bu.edu}
\end{itemize}
\end{itemize}

\item {\bf Lecture Times}: MTW 1:00-3:30PM
\item {\bf Classroom}: CDS 263
\item {\bf Reference Book}:\kern6pt
\begin{itemize}
\item
Modern Compiler Implementation in ML by Andrew W. Appel. ISBN 0-521-
58274-1. Cambridge University Press.
\end{itemize}

\item {\bf Homepage}:
{\tt  https://hwxi.github.io/TEACHING/CS391/2025Sum1}

\item {\bf Description}:
Modern Compiler Construction in Python is a course that introduces
students to some basics in the design and implementation of
compilers. In this course, we teach the theory behind various
components of a compiler as well as the programming techniques
involved to put the theory into practice. In particular, we adopt a
style of modern compiler construction that builds a compiler by
stringing a sequence of translations sharing a common closure-based
interpreter-like structures.  The chosen programming language for
implementation is Python 3. However, you can seek the instructor’s
approval to choose a functional programming language as your
implementation language if you so wish.

\item {\bf Prerequisites}:
This course is designed for students who already have an immediate
level of proficiency in Python. You are expected to be familiar with a
text editor (e.g., vim, emacs, vscode).

\item {\bf Exams}:
\begin{itemize}
\item
First midterm exam: TBA
\item
Second midterm exam: TBA
\item
The final exam date is yet to be anounced.
\end{itemize}

\item {\bf Grades}
The final score is calculated using the following formula.
\[\mbox{final score = 30\%$\cdot$(homework) + 20\%$\cdot$(midterm) + 40\%$\cdot$(final) + 10\%$\cdot$(participation)}\]
The final letter grade is calculated as follows.
\begin{itemize}
\item{\bf A}: final score is $80\%$ or above
\item{\bf B}: final score is $70\%$ or above
\item{\bf C}: final score is $60\%$ or above
\item{\bf D}: final score is $50\%$ or above
\end{itemize}

\item
{\bf Program Submission}:
Programming assignments are to be submitted via the Gradescope system.

\item
{\bf Attendance}:
It is expected that you will attend the lectures.

\item
{\bf Academic Integrity}:
We adhere strictly to the standard BU guidelines for academic
integrity. For this course, it is perfectly acceptable for you to discuss
the general concepts and principles behind an assignment with other
students. However, it is not proper, without prior authorization of the
instructor, to arrive at collective solutions. In such a case, each student
is expected to develop, write up and hand in an individual solution and, in
doing so, gain a sufficient understanding of the problem so as to be able
to explain it adequately to the instructor.  Under {\em no} circumstances
should a student copy, partly or wholly, the completed solution of another
student. If one makes substantial use of certain code that is not written by
oneself, then the person must explicitly mention the source of the involved
code.

\end{itemize}

\end{document}
